% !TEX TS-program = pdflatex
% !TEX encoding = UTF-8 Unicode

% This is a simple template for a LaTeX document using the "article" class.
% See "book", "report", "letter" for other types of document.

\documentclass[11pt]{article} % use larger type; default would be 10pt


\usepackage[utf8]{inputenc} % set input encoding (not needed with XeLaTeX)

%%% Examples of Article customizations
% These packages are optional, depending whether you want the features they provide.
% See the LaTeX Companion or other references for full information.

%%% PAGE DIMENSIONS
\usepackage{geometry} % to change the page dimensions
\geometry{a4paper} % or letterpaper (US) or a5paper or....
% \geometry{margin=2in} % for example, change the margins to 2 inches all round
% \geometry{landscape} % set up the page for landscape
%   read geometry.pdf for detailed page layout information

\usepackage{graphicx} % support the \includegraphics command and options

% \usepackage[parfill]{parskip} % Activate to begin paragraphs with an empty line rather than an indent

%%% PACKAGES
\usepackage{booktabs} % for much better looking tables
\usepackage{array} % for better arrays (eg matrices) in maths
\usepackage{paralist} % very flexible & customisable lists (eg. enumerate/itemize, etc.)
\usepackage{verbatim} % adds environment for commenting out blocks of text & for better verbatim
\usepackage{subfig} % make it possible to include more than one captioned figure/table in a single float
% These packages are all incorporated in the memoir class to one degree or another...

\usepackage{amssymb,amsmath}
\usepackage{amsfonts}

\usepackage{physics}

\usepackage[usenames,dvipsnames,svgnames,table]{xcolor}



%%% HEADERS & FOOTERS
\usepackage{fancyhdr} % This should be set AFTER setting up the page geometry
\pagestyle{fancy} % options: empty , plain , fancy
\renewcommand{\headrulewidth}{0pt} % customise the layout...
\lhead{}\chead{}\rhead{}
\lfoot{}\cfoot{\thepage}\rfoot{}

%%% SECTION TITLE APPEARANCE
\usepackage{sectsty}
\allsectionsfont{\sffamily\mdseries\upshape} % (See the fntguide.pdf for font help)
% (This matches ConTeXt defaults)

%%% ToC (table of contents) APPEARANCE
\usepackage[nottoc,notlof,notlot]{tocbibind} % Put the bibliography in the ToC
\usepackage[titles,subfigure]{tocloft} % Alter the style of the Table of Contents
\renewcommand{\cftsecfont}{\rmfamily\mdseries\upshape}
\renewcommand{\cftsecpagefont}{\rmfamily\mdseries\upshape} % No bold!

\newcommand*{\mathcolor}{}
\def\mathcolor#1#{\mathcoloraux{#1}}
\newcommand*{\mathcoloraux}[3]{%
  \protect\leavevmode
  \begingroup
    \color#1{#2}#3%
  \endgroup
}

\newcommand*\mean[1]{\overline{#1}}
\newcommand{\integral}{\int\limits_{-\infty}^{\infty}}

%%% END Article customizations

%%% The "real" document content comes below...

\title{Question 2}
%\author{The Author}
\date{} % Activate to display a given date or no date (if empty),
         % otherwise the current date is printed

\begin{document}
\maketitle

\section{\(Q_x\) in reprezentarea pozitiilor}
\label{sec:Qx-x}

\begin{align*}
  \mean{\Delta Q_x^2} &= (\mean{Q_x^2}) - (\mean{Q_x})^2 \\
  \mean{Q_x} &= \expval{Q_x}{\Psi} =
    \integral \bra{\Psi}Q_x\ket{x}\braket{x}{\Psi}dx =
    \integral \bra{\Psi}x\ket{x}\braket{x}{\Psi}dx =
    \integral x \abs{\braket{x}{\Psi}}^2 dx \\ &=
    \integral x \abs{\Psi(x)}^2 dx \\
  \mean{Q_x^2} &= \expval{Q_x^2}{\Psi} =
    \integral x^2 \abs{\Psi(x)}^2 dx \\
  \abs{\Psi(x)}^2 &= \frac{\abs{A}^2}{\sqrt{1+\qty(\frac{\hbar t}{m d^2})^2}}
    \exp\qty[- \frac{\qty(x - \frac{\hbar k_0}{m} t)^2}{d^2 \qty[1 + \qty(\frac{\hbar t}{m d^2})^2]}] \\
\end{align*}
\(
  \text{Not. } \alpha = \frac{\abs{A}^2}{\sqrt{1+\qty(\frac{\hbar t}{m d^2})^2}},
    \beta = \frac{\hbar k_0}{m} t, \gamma = d^2 \qty[1 + \qty(\frac{\hbar t}{m d^2})^2]
\)
\begin{align*}
  \abs{\Psi(x)}^2 &= \alpha \exp\qty[- \frac{\qty(x-\beta)^2}{\gamma}] \\
  \mean{Q_x} &= \alpha \integral x \exp\qty[- \frac{\qty(x-\beta)^2}{\gamma}] dx
    = \alpha \beta \sqrt{\gamma \pi} = \frac{\hbar k_0}{m} t \\
  \mean{Q_x^2} &= \alpha \integral x^2 \exp\qty[- \frac{\qty(x-\beta)^2}{\gamma}] dx
    = \alpha \sqrt{\gamma \pi} \qty(\frac{\gamma}{2} + \beta^2)
    = \frac{d^2}{2}\qty[1 + \qty(\frac{\hbar t}{m d^2})^2] + \frac{\hbar^2 k_0^2}{m^2} t^2 \\
  \mean{\Delta Q_x^2} &= \frac{d^2}{2}\qty[1 + \qty(\frac{\hbar t}{m d^2})^2] 
\end{align*}

\section{\(Q_x\) in reprezentarea impulsurilor}
\label{sec:Qx-p}

\begin{align*}
  \mean{Q_x} &= \expval{Q_x}{\Psi} =
    \integral \bra{\Psi}Q_x\ket{p}\braket{p}{\Psi}dp =
    \integral \Phi^*(p) i \hbar \pdv{p} \Phi(p) dp =
    \qty(\mean{Q_x})_0 + \frac{t}{m}\mean{P_x}
\end{align*}

\section{\(\mean{Q_x^2}\)}
\label{sec:Qx2}

\begin{equation}
\label{eq:Qx}
  \mean{Q_x^2} = \expval{Q_x^2}{\Psi} = -\hbar^2 \integral \Phi^*(p) \pdv[2]{p} \Phi(p) dp
\end{equation}

\begin{align*}
  \Phi(p) &= \varphi(p) \exp\qty(- \frac{i p^2}{2\hbar m} t) \\
  \Phi^*(p) &= \varphi^*(p) \exp\qty(\frac{i p^2}{2\hbar m} t) \\
  \pdv{p} \Phi(p) &= \qty(\pdv{\varphi}{p} - \frac{i p}{\hbar m} t \varphi)
    \exp\qty(- \frac{i p^2}{2\hbar m} t) \\
  \pdv[2]{p} \Phi(p) &= \qty(\pdv[2]{\varphi}{p} - \frac{i}{m \hbar} t \varphi
    - 2\frac{i p}{m \hbar} t \pdv{\varphi}{p} - \frac{p^2}{\hbar^2 m^2} t^2 \varphi)
    \exp\qty(- \frac{i p^2}{2\hbar m} t) \\
  \mean{Q_x^2} &= -\hbar^2 \integral \varphi^* \pdv[2]{\varphi}{p}
    - \frac{i}{m \hbar} t \abs{\varphi}^2 - 2\frac{i p}{m \hbar} t \varphi^* \pdv{\varphi}{p}
    - \frac{p^2}{\hbar^2 m^2} t^2 \abs{\varphi}^2 dp \\ &=
  \qty(\mean{Q_x^2})_0
    + \frac{i \hbar}{m} t \integral \qty(\mathcolor{red}{\abs{\varphi}^2 + 2p\varphi* \pdv{\varphi}{p}}) dp
    + \frac{t^2}{m^2} \mean{P_x^2} \\
\end{align*}
Integrand prin parti (\ref{eq:Qx}) obtinem:
\begin{align*}
  \mean{Q_x^2} &= \hbar^2 \integral \pdv{\Phi^*}{p} \pdv{\Phi}{p} dp \\ &=
    \qty(\mean{Q_x^2})_0
    + \frac{i \hbar}{m} t \integral p \qty(\varphi^* \pdv{\varphi}{p} - \pdv{\varphi^*}{p}\varphi)dp
    + \frac{t^2}{m^2} \mean{P_x^2} \\
\end{align*}



{\color{PineGreen} Problema ta cea mai mare e ca vrei sa calculezi media unei observabile si pui in evidenta o parte complexa :) :) :).

Pe scurt, partea imaginara, care nu e OK se prelucreaza astfel:

\begin{align}
\int dp 2p\phi^*(p)\frac{\partial\phi}{\partial p}=\int dp p \frac{\partial}{\partial p}\left(|\phi(p)|^2\right)=-\int dp \frac{\partial p}{\partial p}|\phi(p)|^2
\end{align}
adica ce aveai tu scris cu rosu se anuleaza. Cum ziceam, trebuia sa-ti dea de gandit in primul rand partea imaginara in medie.
In afara de asta, ce am scris aici e in esenta consecinta faptului ca $Q$ este autoadjunct, adica media ta TREBUIE sa poata fi scrisa
\begin{align}
\langle \Psi|Q^2|\Psi\rangle\equiv\langle Q\Psi|Q\Psi\rangle\equiv\int dp {\vline i\hbar\frac{\partial\Phi}{\partial p}\vline}^2\in {\bf R}
\end{align}
}

\end{document}
